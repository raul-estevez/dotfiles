% Template heavily based on Gilles Castel preamble.tex 
% https://github.com/gillescastel/university-setup/blob/master/preamble.tex

% Basic packages
\usepackage[a4paper, margin=2cm]{geometry}
%\usepackage[english, spanish, es-tabla]{babel}
\usepackage{parskip} % Doesnt indent paragraph

% Math
\usepackage{amsfonts, amsmath, amsthm, amssymb}
\usepackage{mathtools}
% fancy closed integrals
\usepackage{esint}
% TODO notes
%\usepackage{todonote}
% Fancy headers
%\usepackage{fancyhdr}
\usepackage{titleps}

% Good looking units with \si{} command
\usepackage{siunitx}

% graphics
\usepackage{graphicx}
%\usepackage[outdir=figures/]{epstopdf}
\usepackage{epstopdf}

% Quotations \say{}
\usepackage{dirtytalk}

% Fancy chapter 
\usepackage[Lenny]{fncychap}
% Chapter config chapter name and page number (adjusted for odd and even pages)
\newpagestyle{main}{
    \setheadrule{.4pt}
    \sethead[\thepage][][\sffamily\thechapter\quad\chaptertitle]{\sffamily\thechapter\quad\chaptertitle}{}{\thepage}
}

% Use subfiles for easy organizing
\usepackage{subfiles}

% \diff for upright (roman) d
\newcommand{\diff}{\mathop{}\!\mathrm{d}}

% \vv as bold font vector
\usepackage{bm}
\newcommand{\vv}[1]{\bm{\mathrm{#1}}}

% \floor and \ceil functions
\DeclarePairedDelimiter\ceil{\lceil}{\rceil}
\DeclarePairedDelimiter\floor{\lfloor}{\rfloor}

% Number equations by section
\counterwithin{equation}{section}

% For box around Definition, Theorem (también es lo de poner colores en los
% teoremas y esas cosas)
%\usepackage{mdframed}
%\mdfsetup{skipabove=0.5em,skipbelow=0.5em}
%\theoremstyle{definition}
%\newmdtheoremenv[nobreak=true]{definition}{\sffamily Definition}
%\newmdtheoremenv[nobreak=true]{theorem}{\sffamily Theorem}
%\newmdtheoremenv[nobreak=true]{note}{\sffamily Note}
%\newmdtheoremenv[nobreak=true]{notation}{\sffamily Notation}
%\newmdtheoremenv[nobreak=true]{example}{\sffamily Example}
%\newmdtheoremenv[nobreak=true]{exercice}{Ejercicio}


% Fancy gray box arround theorems, propositions, examples
% The last {} en each enviorement is the auto label prefix i.e. thm:fundamental_calculus, def:partial_derivative... are all labels 
%\usepackage{tcolorbox}
%\tcbuselibrary{theorems}
%\tcbuselibrary{skins}
%\newtcbtheorem[number within=chapter]{thm}{Theorem}
%{theorem style=change apart,enhanced,arc=0mm,outer arc=0mm,
%    boxrule=0pt,toprule=1pt,bottomrule=0pt,left=0.2cm,right=0.2cm,
%    titlerule=0.5em,toptitle=0.1cm,bottomtitle=-0.1cm,top=0.2cm,
%    colframe=white!25!black,colback=white!80!gray,coltitle=white,
%    title style={white!25!black},
%    fonttitle=\sffamily,fontupper=\normalsize}{thm}
%
%\newtcbtheorem[number within=chapter]{prop}{Proposition}
%{theorem style=change apart,enhanced,arc=0mm,outer arc=0mm,
%    boxrule=0pt,toprule=1pt,bottomrule=0pt,left=0.2cm,right=0.2cm,
%    titlerule=0.5em,toptitle=0.1cm,bottomtitle=-0.1cm,top=0.2cm,
%    colframe=white!25!black,colback=white!80!gray,coltitle=white,
%    title style={white!25!black},
%    fonttitle=\sffamily,fontupper=\normalsize}{prop}
%
%\newtcbtheorem[number within=chapter]{example}{Example}
%{theorem style=change apart,enhanced,arc=0mm,outer arc=0mm,
%    boxrule=0pt,toprule=1pt,bottomrule=0pt,left=0.2cm,right=0.2cm,
%    titlerule=0.5em,toptitle=0.1cm,bottomtitle=-0.1cm,top=0.2cm,
%    colframe=white!25!black,colback=white!80!gray,coltitle=white,
%    title style={white!25!black},
%    fonttitle=\sffamily,fontupper=\normalsize}{ex}
%
%\newtcbtheorem[number within=chapter]{definition}{Definition}
%{theorem style=change apart,enhanced,arc=0mm,outer arc=0mm,
%    boxrule=0pt,toprule=1pt,bottomrule=0pt,left=0.2cm,right=0.2cm,
%    titlerule=0.5em,toptitle=0.1cm,bottomtitle=-0.1cm,top=0.2cm,
%    colframe=white!25!black,colback=white!80!gray,coltitle=white,
%    title style={white!25!black},
%    fonttitle=\sffamily,fontupper=\normalsize}{def}

% Figures from inkscape
\usepackage{import}
\usepackage{xifthen}
\usepackage{pdfpages}
\usepackage{transparent}

\newcommand{\incfig}[2][1]{%
    \def\svgwidth{#1\columnwidth}
    \import{./figures/}{#2.pdf_tex}
}

% Correct spacing and use long arrows in implies and iff
\renewcommand{\implies}{\;\Longrightarrow\;}
\renewcommand{\iff}{\;\Longleftrightarrow\;}

% sgn function
\DeclareMathOperator{\sgn}{sgn}
% My name
\author{Raúl Estévez Gómez}

