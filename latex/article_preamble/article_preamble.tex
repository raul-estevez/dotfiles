% Template heavily based on Gilles Castel preamble.tex 
% https://github.com/gillescastel/university-setup/blob/master/preamble.tex

% Basic packages
\usepackage[a4paper, margin=2cm]{geometry}
%\usepackage[english, spanish, es-tabla]{babel}
\usepackage{parskip} % Doesnt indent paragraph

% Math
\usepackage{amsfonts, amsmath, amsthm, amssymb}
\usepackage{mathtools}
% fancy closed integrals
\usepackage{esint}
% TODO notes
%\usepackage{todonote}
% Fancy headers
%\usepackage{fancyhdr}
\usepackage{titleps}

% \vv for vectors (the [] argument is the arrow type see https://ctan.fisiquimicamente.com/macros/latex/contrib/esvect/esvect.pdf)
\usepackage[e]{esvect}

% Good looking units with \si{} command
\usepackage[per-mode=symbol]{siunitx}
\DeclareSIUnit \belm {Bm} % Declare dBm

% graphics
\usepackage{graphicx}
\graphicspath{ {./figures/} }
%\usepackage[outdir=figures/]{epstopdf}
\usepackage{epstopdf}
\usepackage{caption}
\usepackage{subcaption}

% Quotations \say{}
\usepackage{dirtytalk}

% Fancy gray box arround theorems, propositions, examples
% The last {} en each enviorement is the auto label prefix i.e. thm:fundamental_calculus, def:partial_derivative... are all labels 
%\usepackage{tcolorbox}
%\tcbuselibrary{theorems}
%\tcbuselibrary{skins}
%\newtcbtheorem[number within=section]{thm}{Theorem}
%{theorem style=change apart,enhanced,arc=0mm,outer arc=0mm,
%    boxrule=0pt,toprule=1pt,bottomrule=0pt,left=0.2cm,right=0.2cm,
%    titlerule=0.5em,toptitle=0.1cm,bottomtitle=-0.1cm,top=0.2cm,
%    colframe=white!25!black,colback=white!80!gray,coltitle=white,
%    title style={white!25!black},
%    fonttitle=\sffamily,fontupper=\normalsize}{thm}
%
%\newtcbtheorem[number within=section]{prop}{Proposition}
%{theorem style=change apart,enhanced,arc=0mm,outer arc=0mm,
%    boxrule=0pt,toprule=1pt,bottomrule=0pt,left=0.2cm,right=0.2cm,
%    titlerule=0.5em,toptitle=0.1cm,bottomtitle=-0.1cm,top=0.2cm,
%    colframe=white!25!black,colback=white!80!gray,coltitle=white,
%    title style={white!25!black},
%    fonttitle=\sffamily,fontupper=\normalsize}{prop}
%
%\newtcbtheorem[number within=section]{example}{Example}
%{theorem style=change apart,enhanced,arc=0mm,outer arc=0mm,
%    boxrule=0pt,toprule=1pt,bottomrule=0pt,left=0.2cm,right=0.2cm,
%    titlerule=0.5em,toptitle=0.1cm,bottomtitle=-0.1cm,top=0.2cm,
%    colframe=white!25!black,colback=white!80!gray,coltitle=white,
%    title style={white!25!black},
%    fonttitle=\sffamily,fontupper=\normalsize}{ex}
%
%\newtcbtheorem[number within=section]{definition}{Definition}
%{theorem style=change apart,enhanced,arc=0mm,outer arc=0mm,
%    boxrule=0pt,toprule=1pt,bottomrule=0pt,left=0.2cm,right=0.2cm,
%    titlerule=0.5em,toptitle=0.1cm,bottomtitle=-0.1cm,top=0.2cm,
%    colframe=white!25!black,colback=white!80!gray,coltitle=white,
%    title style={white!25!black},
%    fonttitle=\sffamily,fontupper=\normalsize}{def}

% Figures from inkscape
\usepackage{import}
\usepackage{xifthen}
\usepackage{pdfpages}
\usepackage{transparent}

\newcommand{\incfig}[1]{%
    \import{./figures/}{#1.pdf_tex}
}

% Correct spacing and use long arrows in implies and iff
\renewcommand{\implies}{\;\Longrightarrow\;}
\renewcommand{\iff}{\;\Longleftrightarrow\;}

% sign function
\DeclareMathOperator{\sgn}{sgn}
% sinc function
\DeclareMathOperator{\sinc}{sinc}
% rect function
\DeclareMathOperator{\rect}{rect}


% abs |\cdot| and norm ||\dcot|| operators
\DeclarePairedDelimiter\abs{\lvert}{\rvert}%
\DeclarePairedDelimiter\norm{\lVert}{\rVert}%

% \floor and \ceil functions
\DeclarePairedDelimiter\ceil{\lceil}{\rceil}
\DeclarePairedDelimiter\floor{\lfloor}{\rfloor}

% Number equations by section
\counterwithin{equation}{section}

% \diff for upright (roman) d
\newcommand{\diff}{\mathop{}\!\mathrm{d}}

% \vv as bold font vector
%\usepackage{bm}
%\newcommand{\vv}[1]{\bm{\mathrm{#1}}}

% My name
\author{Raúl Estévez Gómez}

